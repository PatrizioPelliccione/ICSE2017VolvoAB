\section{Conclusions and future works}\label{sec:conclusions}

This work testifies that DbC can be used for improving the robustness of AUTOSAR software components. % for better robustness.
Th approach we proposed in this work %My way in the thesis is 
proposes to build additional pre-condition, post-condition and invariant components around the main processing component. In other words, checks for input, output and invariant are separated from the original component. By testing the components modified in this way, the results prove that DbC improves robustness. This new design of components has strengths as better robustness, better readability, better understandability and low redundancy. It is very easy to apply this way to modify existing software components that are manually coded. % Two points are worth mentioning when applying this way. One is that the 
In order to apply the approach developers should analyse the documented specification or the stakeholders' requirements carefully to know all the possible values of the input, output and invariant. %Another is that the pre-condition, post-condition and invariant components should be well-defined to cover all the values known from the first point. Certainly, the weaknesses such as the possibility of the errors brought by additional components, also exist. That is why we need to go through careful consideration when applying this way of Design by Contract. The Design by Contract approach has been wildly used in many different software applications and tools supported by programming languages themselves or third-party tools. There may be other more effective ways of applying Design by Contract to AUTOSAR software components to improve robustness. That is worthy of further exploration.

%Many AUTOSAR software components are automatically generated from models in modeling tools. For example, some software components in the DEDICATE framework are modeled in Simulink or TargetLink and C source code is generated from TargetLink~\cite{pp}. These components are hard to modify with my way of applying Design by Contract. If the pre-condition, post-condition and invariant components can be modeled in the modeling tools and C source code can also be automatically generated, this will be a great progress.

%Further more, when searching for Design by Contract on the Internet, there are many third-party tools that can be used to support the programming languages that do not have Design by Contract language features. Developing 
%As future work we plan to %replicate the experiment within real
%apply the approach on real environments, i.e. on software components deployed on real trucks, and to make a large scale experiment. 
%it would be valuable to develop a tool that can support C language in AUTOSAR system to directly define pre-conditions, post-conditions and invariants for AUTOSAR software components. % is worth of trying.

%I believe that 
Concluding, DbC has high potential for industrial deployment, however, we need further investigations and evaluation results with more complex and real-life applications, e.g. by considering one complete feature such as cruise control. For example,  we need to better understand the impact that deploying DbC might have on real-time requirements, memory footprint, computational load,  etc. Such investigations will be part of our future work.

%Could be related to the previous comment. For a few number of simpler and smaller components, performance may not be a critical issue and could well be acceptable, considering the gains in robustness. However, we (at least, I) don't know how deployment of DbC scales with various attributes/properties of components. Also, we can think of performance from various aspects. I actually suggested Yulai to consider various factors (memory footprint, execution time, etc.) and perform comparison between component w/o DbC and component with DbC. I am not sure if Yulai has performed such comparisons.